\chapter{Evaluación del desarrollo}

%%%%%%%%%%%%%%%%%%%%%%%%%%%%%%%%%%%%%%%%%%%%%%%55
\section{Dificultades encontradas}

\subsection{Dificultades presentadas por el dominio}
Las dificultades del dominio estuvieron relacionadas a la comprensión no solamente del modelo de arquitectura Q si no también a su \ojo{función}{propósito} didáctic\ojo{a}{o}, ya que el objetivo del simulador no \ojo{era solamente la implementación del lenguaje}{es solamente la simulación de la arquitectura y la ejecución de programas,} sino también el proveer a los alumnos la capacidad de \ojo{realizar cosas}{ejercitar situaciones} conceptualmente erróneas, por lo que se requería una comprensión didáctica del problema más allá de la \ojo{teoría}{especificación} de las arquitecturas Q. 


\subsection{Dificultades de diseño}
En primera instancia se opto por implementar un modelo de objetos que utilizaba un objeto de la clase \textbf{Programa} a lo largo de toda la ejecución \ojo{(procesaba las instrucciones no leyendo de la matriz memoria, si no, pidiendo la siguiente instrucción al objeto instancia de la clase Programa), sobreviviendo así las distintas etapas una vez que fue creado y evitando la creación de un objeto cuya responsabilidad sea interpretar el código máquina alojado en la memoria}{(sacar los paréntesis y detallar mas)}. Luego\ojo{, al caer en la cuenta de}{entendimos} que un programa no sólo podía modificar su entorno al ser ejecutado (otras celdas de memoria que no ocupen su código maquina, celdas de puertos, registros, etc) si no que también podría sobreescribir su código maquina (ya sea con ese propósito o sólo por un \ojo{ConcepError}{error conceptual}), o bien, que el alumno debía tener la posibilidad de seguir ejecutando más allá del código máquina alojado en memoria o más, inevitablemente se \ojo{opto por refactorear todo el}{necesitó corregir gran parte del} modelo agregando una clase denominada \textbf{Intérprete}, cuya responsabilidad es interpretar la siguiente instrucción alojada en memoria para que luego sea ejecutada, y descartando el objeto instancia de \textbf{Programa} una vez que éste es cargado en memoria con éxito.\\

\ojo{Tuvieron que ser solicitadas además extensiones al equipo de desarrolladores de Arena para poder realizar la interfaz con dicho framework.}{(detallar)}

%%%%%%%%%%%%%%%%%%%%%%%%%%%%%%%%%%%%%%%%%%%%%%%55
\section{Casos de prueba}

%%%%%%%%%%%%%%%%%%%%%%%%%%%%%%%%%%%%%%%%%%%%%%%55
\section{Ejemplos de uso}