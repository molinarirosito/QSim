\documentclass[10pt,a4paper]{article}
\usepackage{Qsim}

\author{\susi \and \tati}
\title{\includegraphics[width=4cm]{logoQSim.png}\\Simulador de Arquitecturas Q}


\pagestyle{fancy}
\fancyhead{} % limpio la cabecera
\fancyhead[R]{\nouppercase{\rightmark}}
\fancyhead[L]{\nouppercase{\leftmark}}
\fancyfoot[R]{\includegraphics[width=2cm]{logoQSim.png}}

\begin{document}
\maketitle

\newpage
\tableofcontents

\newpage
%\chapter{Introducción}

\begin{abstract}


Una de las primeras asignaturas que debe recorrer un estudiante de la \tpi\ es \textbf{\orga}. En esta materia los estudiantes descubren los componentes funcionales que conforman un sistema de cómputos, con el fin de comprender un modelo de ejecución de programas que está presente hoy en día en la mayoría de las computadoras personales.
Este trabajo...
\end{abstract}
\part{Contexto}
%%%%%%%%%%%%%%%%%%%%%%%%%%%%%%%%%%%%%%%%%%%%%%%%%%%%%%%%%%%
\section{Sobre la materia \orga (susi)}


Si bien los conceptos fundamentales de la ejecución de programas son independientes de las arquitecturas de computadoras comerciales, es conveniente explicar los mismos dando un marco específico. Por otro lado, las diferentes arquitecturas que subyacen los numerosos modelos disponibles en el mercado, incluyen un basto conjunto de herramientas y recursos de lenguaje para el control del funcionamiento de una computadora, pero que agregan complejidad innecesaria a la comprensión de los conceptos funcionales y la didáctica de la materia.

La propuesta de la asignatura \orga\ es utilizar la arquitectura \Q, una arquitectura \textit{assembly-like} teórica (esto es, que no existe una computadora real que la implemente) basada en el modelo de ejecución de Von Neumann y cuya característica principal es la de ser minimalista y presentarse en 'capas'. Este enfoque permite ir incorporando los conceptos de manera gradual a partir de versiones escalonadas de la arquitectura, denominadas $Q_1$, $Q_2$, $Q_3$, $Q_4$, $Q_5$ y  $Q_6$.\\


Actualmente, la arquitectura se presenta a los estudiantes mediante especificaciones formales que deben analizar y comprender para utilizar el lenguaje de programación y resolver los problemas que se les plantean en la práctica. Entendemos que les resulta de gran utilidad incorporar una herramienta que les permita probar sus ejercicios de una manera automatizada y es por eso que se desarrolló un simulador para esta arquitectura.

%%%%%%%%%%%%%%%%%%%%%%%%%%%%%%%%%%%%%%%%%%%%%%%%%%%%%%%%%%%
\section{Conceptos importantes}

\subsection{Enfoque de Von Neumann}

El matemático John Von Neumann en el año 1945 se encontraba colaborando en el proyecto ENIAC (\textit{Electronic Numerical Integrator And Computer}, primer computadora electrónica de propósito general, diseñada para ser utilizada por el ejército norteamericano. La ENIAC podía ser programada para realizar operaciones complejas e incluso decisiones, iteraciones y subrutinas, pero la tarea de resolver un problema y volcarlo en la máquina era tan complejo que podía tomar semanas. Luego que el programa era diseñado en papel, el proceso de representarlo en la máquina ENIAC mediante la manipulación de cables e interruptores tomaba varios días. Entonces Von Neumann se comenzó a interesar por la problemática que significaba la la necesidad de reconfigurar la máquina para cada nueva tarea y tan sólo cuatro años más tarde propone y desarrolla una solución a este problema que se basaba en almacenar la información sobre las operaciones a realizar en la misma memoria utilizada para los datos, a partir de su codificación en código binario al igual que los datos.\\

Además, este enfoque generaliza la organización de las computadoras distinguiendo en tres partes interconectadas: La CPU (con la unidad aritmético-lógica o ALU y la unidad de control) la memoria, y un módulo de entrada/salida. La interconexión es llevada a cabo por un bus de sistema que proporciona un medio de transporte de los datos entre las distintas partes. \\

Con la propuesta de este modelo, Von Neumann incorpora el concepto de \textbf{programa almacenado} en memoria. Con esta idea, el programa se codifica de cierta manera para que pueda ser almacenado en memoria principal y posteriormente pueda ser ejecutado quizás múltiples veces. De esta manera, la lógica del programa puede ser ''recordada'' y el programa toma un valor mayor, a diferencia de lo que ocurría hasta entonces, donde el programa se reflejaba en un conjunto de configuraciones de cables aplicadas a los equipos. Esto implica una separación entre el mecanismo de ejecución (el \textit{hardware}) y la lógica de computo o instrucciones (el \textit{software}). La codificación en binario de las instrucciones de un programa se denomina \textbf{\codmaq}.\\

Por otro lado este tipo de diseño, que permite un programa almacenado, también da la posibilidad de que la ejecución de las instrucciones modifique el código máquina del mismo u otro programa. Por ejemplo un programa podría modificar o incrementar las referencias a las direcciones de memoria que tenga en algunas instrucciones y luego volver a ejecutar dichas instrucciones con el fin de procesar celdas diferentes de memoria cada vez. Esta característica es potente pero presenta un alto riesgo pues las modificaciones en los programas podía ser algo perjudicial, por accidente o por diseño.

\subsection{Organización de la computadora}

La CPU (Unidad Central de Procesamiento del inglés: Central Processing Unit), es el componente principal y el encargado ejecutar los programas y procesar los datos. La CPU contiene otros componentes de importancia tales como la \UC, algunos registros de uso específico como el contador de programa (PC o \textit{Program Counter}), el registro de instrucción (IR - \textit{Instruction Register}) y el \SP\ (SP - \textit{Stack pointer}), otros registros de uso general y la Unidad Aritmético-Lógica (ALU).\\

La \UC\ dirige el ciclo de ejecución de cada instrucción, pidiendo la lectura de celdas de memoria donde esta alojada, decodificándola  y ejecutándola luego en colaboración con los otros componentes del sistema: si es una operación lógica o aritmética le ordena a la ALU su ejecución, si es de movimiento de datos colabora con la memoria ó el módulo de Entrada/Salida.\\

Entre los registros de uso específico, los más importantes son el \textbf{\PC} y el \textbf{\IR}. El \PC\ es un registro que indica la posición de memoria donde estará la siguiente instrucción que debe ejecutarse. Luego de completar el ciclo de ejecución de una instrucción, el PC se incrementa en función de la cantidad de celdas que ocupa el código máquina de esta. El registro de instrucción contiene el código máquina de la instrucción actual una vez que la misma es leída de memoria para luego decodificarla y ejecutarla.\verde{Falta hablar del SP y los flagsssss}{}\\

El diseño de cada arquitectura ofrece un conjunto diferente de registros de uso general para ser usados en los programas. Estos registros son elementos de memoria de alta velocidad y poca capacidad que pueden ser utilizados como variables en los programas. Es importante marcar que pueden almacenar tanto datos como direcciones de memoria.\\

La \ALU, recibe su nombre de las siglas en inglés de \textit{Arithmetic and Logic Unit}. La ALU es un circuito digital que lleva a cabo operaciones aritméticas (suma, resta, multiplicación, división) y las operaciones lógicas como la negación, disyunción, conjunción, etc, entre dos cadenas binarias que son interpretadas como números o valores lógicos.\\

La memoria es un conjunto de celdas numeradas. La numeración de cada celda la identifica inequívocamente por lo cual a esta numeración se le llama dirección. En cada celda de la memoria se pueden almacenar datos o instrucciones en forma de cadenas binarias y este contenido puede leerse y modificarse. En la memoria es donde se alojan los programas que luego serán ejecutados.\\

El bus de sistema es el encargado de transferir los datos entre los componentes de la computadora..La unidad de control al pedir un contenido de una dirección de memoria lo hace a través del bus, y similarmente mismo cuando desea escribir en memoria, y lo hace a partir de que la \UC\ pide la lectura o escritura de celdas de memoria o puertos de entrada/salida.

\subsection{Ejecución de un programa}

La función de una computadora es la ejecución de programas. Los programas se encuentran almacenados en memoria y consisten en una secuencia de instrucciones y es la \UC\ es quien se encarga de ejecutar dichas instrucciones implementando un \textbf{\ciclo}. Para ser almacenadas en memoria, las instrucciones deben codificarse en cadenas binarias (secuencias de ceros y unos) que no son legibles para las personas pero son tales que la \UC\ las puede interpretar y traducir en acciones. Por eso para saber de qué instrucción se trata, y cuáles son los valores o variables (celdas de memoria o registros) que estan involucrados, la \UC\ toma el código máquina de la instrucción y verifica los códigos de operación y modos de direccionamiento. La ejecución de instrucciones se divide en tres etapas importantes: 

\begin{enumerate}
\item \BI
\item \DI
\item \EI
\end{enumerate}
\graf[7cm]{ciclo}{\ciclo}


Al principio de cada \ciclo se lleva a cabo la \BI\ durante la cual se leen las celdas que contienen el \codmaq\ de la instrucción y para lo cual se utiliza el \PC\ que mantiene la dirección de la siguiente instrucción a ejecutar.
El código máquina de la instrucción leída que está en la forma de cadena binaria se carga dentro de otro registro de la CPU, llamado registro de instrucción (IR).\\

Durante la \DI, la \UC\ determina que operación se debe llevar a cabo, y finalmente durante la \EI la \UC realiza el efecto esperado para esa operación, buscando los operandos y modificando la memoria o los registros como resultado final y el ciclo vuelve a comenzar

%%%%%%%%%%%%%%%%%%%%%%%%%%%%%%%%%%%%%%%%%%%%%%%%%%%%%%%%%%%
\section{Estado del arte}

A través de los años de la carrera \tpi, en la materia \orga\ se analizaron distintos enfoques y herramientas para desarrollar los conceptos relacionados a la ejecución de programas en una computadora.\\

Inicialmente se utilizó un simulador de código abierto para la arquitectura Intel 8085, que ofrecía una funcionalidad bastante completa, pero una interfaz que no resultaba del todo intuitiva.
Este simulador se eligió por tratarse de un lenguaje assembler mas reducido, con un repertorio de instrucciones y modos de direccionamiento mas pequeño que contaba con un entorno de prueba (el simulador propiamente dicho) para facilitar la didáctica de la programación en lenguaje ensamblador, pero posteriormente se entendió que las características de la arquitectura no eran las adecuadas para la enseñanza de los contenidos y se descartó.\\
 
Entonces el equipo docente eligió definir una arquitectura teórica que proveyera solamente lo necesario para cumplir con los objetivos de la materia e inspirados en un caso similar de la Universodad de Buenos Aires, se definió la arquitectura \textbf{QARQ}, muy similar a la que se presenta en la sección \ref{apendiceQ}\\


Posteriormente, el equipo docente propuso un cambio en la secuencia didáctica que requirió la división de la especificacion de la arquitectura QARQ en varias partes, donde cada una recibe el nombre de \textbf{Arquitectura Qi} y agrega una nueva funcionalidad (instrucciones o modos de direccionamientos) a la versión anterior, son como capas de cebolla. Sin embargo todos los ejercicios de las arquitecturas Qi se siguen haciendo en papel. \\

Actualmente se busca incorporar el \textbf{Simulador Qsim} en la materia con el objetivo de que los alumnos puedan visualizar el funcionamiento de una computadora al mas mínimo detalle, a través de la ejercitación del \ciclo.

%%%%%%%%%%%%%%%%%%%%%%%%%%%%%%%%%%%%%%%%%%%%%%%%%%%%%%%%%%%
\section{Arquitecturas Q (Susi)}

Las versiones de la arquitectura Q están pensadas para incorporar funcionalidades de manera que la curva de aprendizaje sea adecuada para los alumnos, siendo paulatina e incremental, es decir, cada arquitectura Qi agrega más funcionalidad (ya sean instrucciones nuevas o modos de direccionamiento) a las arquitecturas Qi anteriores.

\ojo{Incorporar lo que sigue a un gráfico de la estructura en cebolla}{}

%%%%%%%%%%%%%%%%%%%%%%%%%%%%%%%%%%%%%%%%%%%%%%%%%%%%%%%%%%%
\subsection{Q1}


\textbf{Modos de direccionamiento}
\begin{enumerate}
\item Inmediato
\item Registro
\end{enumerate}

\textbf{Instrucciones}
\begin{enumerate}
\item MOV
\item SUB 
\item DIV 
\item ADD 
\item MUL
\end{enumerate}


%%%%%%%%%%%%%%%%%%%%%%%%%%%%%%%%%%%%%%%%%%%%%%%%%%%%%%%%%%%
\subsection{Q2}

\textbf{Modos de direccionamiento}
\begin{enumerate}
\item Modos de direccionamiento \textbf{Q1}
\item Directo 
\end{enumerate}

\textbf{Instrucciones}
\begin{enumerate}
\item Instrucciones \textbf{Q1}
\end{enumerate}

%%%%%%%%%%%%%%%%%%%%%%%%%%%%%%%%%%%%%%%%%%%%%%%%%%%%%%%%%%%
\subsection{Q3}

\textbf{Modos de direccionamiento}
\begin{enumerate}
\item Modos de direccionamiento \textbf{Q2}
\end{enumerate}

\textbf{Instrucciones}
\begin{enumerate}
\item Instrucciones \textbf{Q2}
\item CALL
\item RET
\end{enumerate}

%%%%%%%%%%%%%%%%%%%%%%%%%%%%%%%%%%%%%%%%%%%%%%%%%%%%%%%%%%%
\subsection{Q4}

\textbf{Modos de direccionamiento}
\begin{enumerate}
\item Modos de direccionamiento \textbf{Q3}
\end{enumerate}

\textbf{Instrucciones}
\begin{enumerate}
\item Instrucciones \textbf{Q3}
\item CMP
\item JMP
\item JE 
\item JNE 
\item JLE 
\item JG 
\item JL 
\item JGE 
\item JLEU 
\item JGU 
\item JCS 
\item JNEG 
\item JVS
\end{enumerate}

%%%%%%%%%%%%%%%%%%%%%%%%%%%%%%%%%%%%%%%%%%%%%%%%%%%%%%%%%%%
\subsection{Q5}

\textbf{Modos de direccionamiento}
\begin{enumerate}
\item Modos de direccionamiento \textbf{Q4}
\item Idirecto
\item RegistroIdirecto
\end{enumerate}

\textbf{Instrucciones}
\begin{enumerate}
\item Instrucciones \textbf{Q4}
\end{enumerate}

%%%%%%%%%%%%%%%%%%%%%%%%%%%%%%%%%%%%%%%%%%%%%%%%%%%%%%%%%%%
\subsection{Q6}

\textbf{Modos de direccionamiento}
\begin{enumerate}
\item Modos de direccionamiento \textbf{Q5}
\end{enumerate}

\textbf{Instrucciones}
\begin{enumerate}
\item Instrucciones \textbf{Q5}
\item AND
\item OR
\item NOT
\end{enumerate}



\part{Simulador \qsim}


El simulador desarrollado es una herramienta de utilización sencilla ya que se asume que los alumnos de esta materia están en la etapa inicial de la carrera y se pretende transmitir los conceptos sin distraerlos con detalles de uso y configuración.

\section{Funcionalidad del simulador}

La funcionalidad del simulador puede caracterizarse mediante las siguientes partes importantes:

\begin{itemize}
\item Chequeo de sintaxis de los programas escritos en el lenguaje Q
\item Ensamblado del código fuente de un programa en su correspondiente código máquina
\item Cargado en memoria del código máquina
\item Ejecución paso a paso de un programa cargado en memoria
\end{itemize}

\subsection{Chequeo de sintaxis}\label{parser}

El simulador provee al alumno de un editor de texto en el cual escribirá el programa en un lenguaje Qi, que desea cargar en memoria y ejecutar.
Una vez que el usuario haya terminado la escritura, al momento de cargar el programa, el simulador utilizará un \textit{parser}\footnote{\verde{Definir parser en esta nota al pie}{}} para detectar errores de sintaxis, tales como la falta de una coma o un corchete, o la presencia de símbolos que no pertenecen al lenguaje (como por ejemplo signos de pregunta y símbolos matemáticos); o bien errores semánticos como la combinación incorrecta de elementos del lenguaje, por ejemplo: modos de direccionamiento mal ubicados.
El parser solo revisará lo escrito por el alumno y de acuerdo a las gramática del lenguaje, mostrará alguno de los siguientes estados:

\begin{description}
\item[OK] Este mensaje se obtiene cuando no hubo ningún error de sintaxis. Si se da este resultado, es posible continuar con el ensamblado y cargado en memoria.
\item[SyntaxError] Este mensaje de error se obtiene cuando en alguna línea del programa se detectó algún error de sintaxis o de semántica, como se describió arriba. Cuando ocurre este error se lo acompaña con una descripción lo mas detallada posible para que el alumno detecte donde ocurrió y pueda corregirlo. Un programa con errores no puede ser ensamblado y cargado en memoria.
\end{description}

\subsection{Ensamblado}

Una vez que el programa es sintácticamente válido es posible traducir el código fuente del programa en código máquina (representado en cadenas binarias). Para esto se respeta un formato de instrucción que indica cómo se codifica cada operación y los operandos. \\
Mas detalle al respecto de este proceso en el apéndice \ref{apendiceQ}.

\subsection{Cargado en memoria}

Una vez ensamblado, la representación binaria (o código máquina) del programa será cargado en memoria a partir de una ubicación (celda de memoria) que el alumno puede elegir. Esto permite visualizar el contenido de la memoria (con el programa cargado) y el estado de los registros de la CPU. \verde{La decodificación con desensamblado permite al alumno experimentar otros escenarios y efectos laterales, entre los cuales podemos enumerar:

\begin{itemize}
\item Si la ejecución paso a paso excede los límites del programa, pueden tomarse instrucciones de otra rutina y procesarse como una nueva instruccion.
\item Si en cambio, se intenta ejecutar el contenido de una celda con datos (y no una instrucción) podrá ocurrir que se encuentre una instrucción invalida (por ejemplo, una combinación incorrecta de modos de direccionamiento y códigos de operación) y el alumno verá el mensaje de error pertinente.
\end{itemize}}{Esto me parece que no corresponde aca, sino mas adelante}

Durante la carga del programa en memoria puede ocurrir que el programa no cuenta con el espacio suficiente a partir de la ubicación elegida ya que ocupa más celdas que las que se encuentran disponibles, ya que como se especifica en el apéndice \ref{apendiceQ}, la memoria disponible tiene un tamaño limitado y por este motivo la alocación en memoria del código máquina puede exceder el espacio disponible a partir de la celda inicial anteriormente elegida. Si por el contrario, no se produce este error, el alumno podrá ver el programa cargado en memoria exitosamente.

\subsection{Ejecución paso a paso}

Se provee la funcionalidad de la ejecución paso a paso ya que se desea que el alumno pueda experimentar y así comprender los pasos del ciclo de ejecución. Además puede ejercitarse situaciones que se denominan ''errores conceptuales de programación'' Esto es a lo que llamamos Errores conceptuales, entre los cuales es posible mencionar:

\begin{itemize}
\item Tomar un dato de un sector de memoria equivocado.
\item Que el programa sobrescriba su mismo código máquina.
\item Permitir que la ejecución continue una vez procesadas las instrucciones del programa cargado en memoria.
\end{itemize}

El paso a paso que provee el simulador consiste en las siguientes etapas pertenecientes al \ciclo:

\begin{enumerate}
\item \textbf{Búsqueda de instrucción:} El alumno podrá visualizar el valor que contiene PC (Program counter) donde se encuentra la dirección de la celda en memoria que contiene la próxima instrucción a ejecutar (por ejemplo, en caso de ser la primer instrucción del programa recién cargado, el pc tendrá la dirección de memoria elegida por el alumno para iniciar el cargado del programa en memoria). El simulador, toma de la memoria el código maquina correspondiente a la instrucción que comienza en esa dirección tomada de PC (una instrucción puede ocupar más de una celda de memoria) y los guarda en el \IR\ (\textit{Instruction Register}). Será observable también para el alumno el incremento del registro PC, tantas como celdas ocupe la instrucción actual, lo que conceptualmente es, preparar el contexto de ejecución para tomar la siguiente instrucción.

\item  \textbf{Decodificación:}
En la decodificación el Interprete se encarga de desensamblar el código máquina (abreviado en hexadecimal) que ya fue ubicado en el \IR para mostrar el código fuente de la instrucción actual con sus respectivos operandos. Si el programa escrito por el alumno es sintacticamente y conceptualmente correcto, este paso le permite comprobar que la instrucción actual es la que él mismo escribió y no otra, visualizandola en pantalla. En esta etapa se provee también la oportunidad de que el alumno aprecie otros conceptos, tales como los errores conceptuales mencionados antes.

\item  \textbf{Ejecución}
El execute ejecuta los efectos de la instrucción y muestra en pantalla los cambios en el estado de ejecución: memoria, puertos, registros y flags. Dentro de esta misma etapa se lleva a cabo el almacenamiento de resultados que, cuando sea necesario, guardará el valor resultante de la operación descripta por la instrucción en el operando destino. Esto cambiará el valor de una celda de memoria o de un registro y será visto en pantalla por el alumno.
\end{enumerate}


%%%%%%%%%%%%%%%%%%%%%%%%%%%%%%%%%%%%%%%%%%%%%%%%%%%%%%%%%%%5
\section{Implementación}

%%%%%%%%%%%%%%%%%%%%%%%%%%%%%%%%%%%%%%%%%%%%%%%%%%%%%%%%%%%
\subsection{Tecnología utilizada}
En la presente sección se indica la tecnología elegida para la implementación del simulador, justificando dichas elecciones en cada caso.

\begin{itemize}


\item  \textbf{Lenguaje Scala.}
Elegimos el lenguaje Scala para realizar el simulador ya que durante las cursadas de las asignaturas de TPI no tuvimos la oportunidad de profundizar el dominio de este lenguaje ni aprovechar las ventajas que este ofrecía al combinar el manejo de objetos y las características de un lenguaje funcional. Es por ello que descartamos la elección de un lenguaje con el que estábamos mas familiarizadas, como por ejemplo Java.

\item  \textbf{Framework Arena.}
Utilizamos el framework Arena para realizar la interfaz de usuario del simulador porque es una herramienta de código abierto que tuvimos la oportunidad de conocer en la materia \ui. Al poder ser combinado con \textbf{Scala} nos pareció una buena oportunidad de explotar lo que nos ofrecía para simplificar la definición de la interfaz de usuario, permitiéndonos así enforcarnos en la implementación del modelo. 

\item  \textbf{Eclipse.}
Se eligió utilizar el entorno de programación Eclipse ya que es una herramienta multiplataforma, lo que nos permitió trabajar en diferentes sistemas operativos y con la cual estábamos familizadas.  Por otro lado, la comunidad provee pluggins para manejar proyectos para Scala.

\item  \textbf{Git}
Elegimos git como repositorio externo para trabajar colaborativamente durante el desarrollo, dado que es una herramienta que muy extendida en los desarrollos de software libre.

\end{itemize}

%%%%%%%%%%%%%%%%%%%%%%%%%%%%%%%%%%%%%%%%%%%%%%%%%%%%%%%%%%%
\subsection{Diseño Orientado a Objetos}

\subsubsection{ALU}
Como se observa en la figura \ref{ALU} la ALU tiene toda la responsabilidad en la ejecución de operaciones matemáticas y lógicas, además del cómputo de los flags luego de cada operación. 

\graf{ALU}{Diagrama de clase de la \ALU}
%%%%%%%%%%%%%%%%%%%%%%%%%%%%%%%%%%%%%%%%

\subsubsection{Bus de entrada y salida, memoria y puertos}
Como se observa en la figura \ref{BusEntradaSalida_Memoria_CeldasPuertos} el Bus de entrada y salida tiene la responsabilidad de derivar según donde corresponda (Memoria o Puertos) la escritura o lectura de un dato. Para ello conoce a una instancia de la clase Memoria y a otra de la clase CeldasPuertos. 
Ambas clases conocen una colección de instancias de la clase Celda, y cada Celda a su vez conoce un dato: una instancia de la clase W16 que representa al dato almacenado en una celda de memoria o en un puerto.  

\graf{BusEntradaSalida_Memoria_CeldasPuertos}{Diagrama de las clases del la BusEntradaSalida, Memoria, CeldasPuertos y Celda}
%%%%%%%%%%%%%%%%%%%%%%%%%%%%%%%%%%%%%%%%

\subsubsection{CPU}
Como se observa en la figura \ref{CPU}, la CPU conoce a la ALU, contiene los registros IR y PC, los flags (V,Z,C,N) y los ocho registros de uso general (R0...R7). La responsabilidad de la CPU es actualizar los flags, los registros, actualizar el PC y el IR, y ser la conexión con la ALU.

\graf{CPU}{Diagrama de clase de la CPU}
%%%%%%%%%%%%%%%%%%%%%%%%%%%%%%%%%%%%%%%%

\subsubsection{Intérprete}
Como se observa en la figura \ref{Interprete} el Interprete es un \textit{singleton} que tiene la entera responsabilidad de construir un objeto que representa una instrucción determinada a partir de la decodificación de las celdas de memoria que ocupa su código máquina. Se ocupa de la decodificación de la instrucción.

\graf{Interprete}{Diagrama de clase del Interprete}
%%%%%%%%%%%%%%%%%%%%%%%%%%%%%%%%%%%%%%%%

\subsubsection{Modos de direccionamiento y W16}

\ojo{}{Esta jerarquía de clases permite controlar el acceso a los operandos, mediante un objeto ModoDireccionamiento que controla el acceso a un objeto W16, que es quien contiene el valor propiamente dicho}. 
Como se observa en la figura \ref{ModoDireccionamiento} los modos de direccionamiento extienden del trait ModoDireccionamiento, donde se encuentran declarados mensajes necesarios para manejar todas las subclases de manera polimórfica. Entre estos mensajes podemos enumerar:
\begin{itemize}
\item  \textbf{representacionString:}
Devuelve la representación en string como código fuente, por ejemplo la representación de un \ojo{}{objeto?} \textbf{ADD(R0,R7)}, sería: \code{ADD R0, R7}\ojo{}{(aclarar) este mensaje se utiliza en la etapa de desensamblado}
\item  \textbf{codigo:}
Retorna el string que representa al código del modo de direccionamiento, por el ejemplo, el código de modo de direccionamiento del R7 es \code{100111}. \ojo{}{(aclarar) este mensaje se utiliza en la etapa de ensamblado}
\item  \textbf{getValorString:}
Retorna el dato almacenado en el operando. En el caso de un Inmediato que sea \code{FF56}, devolverá el string "FF56", y en el caso de cualquier registro, retornara el valor que represente su W16.
\end{itemize}


La clase W16 que también esta en la figura \ref{ModoDireccionamiento}, representa el dato que es guardado en memoria. Tiene la \ojo{responsabilidad}{capacidad} de incrementarse, decrementarse, sumar una entero, devolver su representación binaria y su valor en decimal.\\

Los modos de direccionamiento diferentes a Inmediato y Registro, conocen otro modo de direccionamiento que encapsula al objeto \textbf{W16} según corresponda, es decir:
\begin{itemize}
\item \textbf{RegistroIndirecto} conoce una instancia de Registro.
\item \textbf{Directo} conoce conoce una instancia de Inmediato.
\item \textbf{Indirecto} conoce conoce una instancia de Directo.
\end{itemize}
\ojo{Esto se implemento de esta manera para que el leer datos de memoria, puertos o registros, o guardarlos en los mismos sea más sencillo ya que se delega en el modo de direccionamiento que conoce.}{(esto lo sacaría)}

La clase \textbf{Etiqueta} representa las etiquetas creadas por el alumno cuando realiza el programa. Cuando el mismo es cargado en memoria, en función de cual sea la celda de inicio y cuanto ocupen las instrucciones, se calcula la dirección de memoria a la que hace referencia y luego se la descarta reemplazándola por un modo de direccionamiento Inmediato.

\graf[15cm]{ModoDireccionamiento}{Diagrama de clase de la jerarquía de los modos de direccionamiento}
%%%%%%%%%%%%%%%%%%%%%%%%%%%%%%%%%%%%%%%%

\subsubsection{Instrucciones}
Como se observa en la figura \ref{Instruccion} las Instrucciones están \ojo{modeladas en jerarquias que se dividen en}{jerarquizadas en}:

\begin{itemize}
\item Instrucciones de un operando.
\item Instrucciones de dos operandos.
\item Instrucciones de sin operandos.
\item Saltos condicionales.
\end{itemize}

Se realizó dicha jerarquía para permitir la fácil \ojo{agregación}{incorporación} de nuevas instrucciones \ojo{siendo}{como} subclases de la clase que corresponda ya que \ojo{comparten}{de ese modo se reutiliza} comportamiento tal como la manera de mostrarse (en términos de código fuente) y de codificarse (en términos de \codmaq).

\graf{Instruccion}{Diagrama de clase de la Instrucción}

\begin{description}
\item[Instrucciones sin operandos] Como se observa en la figura \ref{Instruccion_SinOperandos} la única instrucción sin operandos implementada en la arquitectura Q es la instrucción \code{RET}. A pesar de esto, se eligió hacer una jerarquía para que luego se facilite \ojo{la inserción al modelo de nuevas instrucciones sin operandos}{la escalabilidad del modelo, permitiendo la inserción de nuevas instrucciones sin operandos}.

\graf[4cm]{Instruccion_SinOperandos}{Detalle de clase Instruccion\_SinOperandos}

\item[Instrucciones de un operando] 
Como se observa en la figura \ref{Instruccion_UnOperando} y \ojo{concordando en lo}{siguiendo con el criterio} mencionado antes sobre la jerarquía de instrucciones, puede haber dos tipos de instrucciones de un operando:

\begin{itemize}
\item Un operando origen.
\item Un operando destino.
\end{itemize}

Ambas clases de instrucciones tienen un solo operando. La diferencia entre ellas es la lógica de ejecución y la manera en la que se ensamblan y desensamblan ya que sus formato de instrucción difiere\ojo{ en donde esta colocado el relleno}{(ver apéndice \ref{apendiceQ})}.

\graf[15cm]{Instruccion_UnOperando}{Diagrama de clase de la Instruccion\_UnOperando}


\item[Instrucciones de dos operandos]

Como se observa en la figura \ref{Instruccion_DosOperandos} la jerarquía de clases de instrucciones de dos operandos es la más amplia por tener mayor cantidad de instrucciones. Todas comparten \ojo{la manera de }{el comportamiento para la} decodificación e interpretación, además de la lógica de impresión.

\graf[15cm]{Instruccion_DosOperandos}{Diagrama de clase de la Instruccion\_DosOperandos}

\end{description}



%%%%%%%%%%%%%%%%%%%%%%%%%%%%%%%%%%%%%%%%

\subsubsection{Programa}

Como se observa en la figura \ref{Programa} la clase Programa conoce un grupo de Instrucciones (las instrucciones que lo componen). Las instancias son creadas por el \textit{parser} (ver sección \ref{parser}), luego se calculan las etiquetas (si es que las tiene) y finalmente cuando es cargado en memoria la instancia de programa es desechada ya que no vuelve a usarse en ningún momento de la ejecución.

\graf{Programa}{Diagrama de clase de la Programa}
%%%%%%%%%%%%%%%%%%%%%%%%%%%%%%%%%%%%%%%%

\subsubsection{Simulador}

Esta e la clase principal del modelo y la encargada de coordinar la ejecución del programa paso a paso. Como se observa en la figura \ref{simulador} la clase \textbf{Simulador} conoce a una instancia de la clase \textbf{CPU}, a una instancia de la clase \textbf{BusEntradaSalida} y a una intancia de la clase \textbf{Instruccion}, que representa a la instrucción que se esta ejecutando en ese momento.\\

La clase Simulador tiene la responsabilidad de obtener el \codmaq\ de la siguiente instrucción, colaborando con el Interprete (ver \ref{interprete}), calcular las etiquetas de un programa, cargar el programa en memoria y los datos en registros,  ejecutar las instrucciones o delegar su ejecución al objeto \textbf{ALU} que conoce a través de la \textbf{CPU} según corresponda, y guardar datos en memoria o registros (almacenamiento de resultados).

\graf[15cm]{simulador}{Diagrama de clase de la Simulador}
%%%%%%%%%%%%%%%%%%%%%%%%%%%%%%%%%%%%%%%%

%Como se observa en la figura \ref{}

%\graf{}{Diagrama de clase de la \}
%%%%%%%%%%%%%%%%%%%%%%%%%%%%%%%%%%%%%%%%

%[width=0.7\textwidth]




\chapter{Evaluación del desarrollo}

%%%%%%%%%%%%%%%%%%%%%%%%%%%%%%%%%%%%%%%%%%%%%%%55
\section{Dificultades encontradas}

\subsection{Dificultades presentadas por el dominio}
Las dificultades del dominio estuvieron relacionadas a la comprensión no solamente del modelo de arquitectura Q si no también a su \ojo{función}{propósito} didáctic\ojo{a}{o}, ya que el objetivo del simulador no \ojo{era solamente la implementación del lenguaje}{es solamente la simulación de la arquitectura y la ejecución de programas,} sino también el proveer a los alumnos la capacidad de \ojo{realizar cosas}{ejercitar situaciones} conceptualmente erróneas, por lo que se requería una comprensión didáctica del problema más allá de la \ojo{teoría}{especificación} de las arquitecturas Q. 


\subsection{Dificultades de diseño}
En primera instancia se opto por implementar un modelo de objetos que utilizaba un objeto de la clase \textbf{Programa} a lo largo de toda la ejecución \ojo{(procesaba las instrucciones no leyendo de la matriz memoria, si no, pidiendo la siguiente instrucción al objeto instancia de la clase Programa), sobreviviendo así las distintas etapas una vez que fue creado y evitando la creación de un objeto cuya responsabilidad sea interpretar el código máquina alojado en la memoria}{(sacar los paréntesis y detallar mas)}. Luego\ojo{, al caer en la cuenta de}{entendimos} que un programa no sólo podía modificar su entorno al ser ejecutado (otras celdas de memoria que no ocupen su código maquina, celdas de puertos, registros, etc) si no que también podría sobreescribir su código maquina (ya sea con ese propósito o sólo por un \ojo{ConcepError}{error conceptual}), o bien, que el alumno debía tener la posibilidad de seguir ejecutando más allá del código máquina alojado en memoria o más, inevitablemente se \ojo{opto por refactorear todo el}{necesitó corregir gran parte del} modelo agregando una clase denominada \textbf{Intérprete}, cuya responsabilidad es interpretar la siguiente instrucción alojada en memoria para que luego sea ejecutada, y descartando el objeto instancia de \textbf{Programa} una vez que éste es cargado en memoria con éxito.\\

\ojo{Tuvieron que ser solicitadas además extensiones al equipo de desarrolladores de Arena para poder realizar la interfaz con dicho framework.}{(detallar)}

%%%%%%%%%%%%%%%%%%%%%%%%%%%%%%%%%%%%%%%%%%%%%%%55
\section{Casos de prueba}

%%%%%%%%%%%%%%%%%%%%%%%%%%%%%%%%%%%%%%%%%%%%%%%55
\section{Ejemplos de uso}

\appendix
\part{Apéndices}
\section{Especificación de la arquitectura Q}\label{apendiceQ}

%%%%%%%%%%%%%%%%%%%%%%%%%%%%%%%%%%%%%%%%%%%%%%%%%%%%%%%%%%%
\subsection{Características generales} 

La Arquitectura Q tiene 8 registros de uso general de 16 bits, denominados R0..R7, registros especiales de 16 bits tales como PC - \textit{Program counter}, SP\footnote{Comienza en la dirección FFEF.} - \textit{Stack Pointer} y los Flags de un bit como Negative, oVerflow, Carry, Zero. También tiene un conjunto de instrucciones que detallaran mas adelante. Todas las instrucciones Alteran los flags excepto MOV, CALL, RET, JMP, Jxx. De las instrucciones que alteran los Flags, todas dejan C y V en 0 a excepción de ADD, SUB y CMP.  
\ojo{La instrucción DIV tiene como efecto [destino (- destino \%\footnote{El carácter \% denota el cociente de la división entera.} origen ]. 
La instrucción MUL\footnote{El resultado de la operación \textbf{MUL} ocupa 32 bits, almacenándose los 16 bits menos significativos en el operando destino y los 16 bits mas significativos en el registro \textbf{R7}.} tiene como efecto [destino (- destino * origen].}{esto va mas adelante!}
Por ultimo tiene una Memoria que tiene direcciones de 16 bit, donde el tamaño de cada celda también es de 16 bit. La Memoria tiene un tamanio de 65536 celdas. \\ 
%%%%%%%%%%%%%%%%%%%%%%%%%%%%%%%%%%%%%%%%%%%%%%%%%%%%%%%%%%%
\subsection{Modos de direccionamiento}
Los siguientes son los modos de direccionamiento implementados en la Arquitectura Q.
\begin{enumerate}

\item \textbf{Inmediato} Representa un operando que denota un valor constante. Es importante notar que este modo direccionamiento es admitido en el operando origen pero no el operando destino. La codificación de este modo se indica en la tabla \ref{tablamodos}.

Ejemplos:
\begin{itemize}
\item \textbf{0x0000} denota el modo de direccionamiento inmediato cuyo valor es cero.
\item \textbf{0x000F} denota el modo de direccionamiento inmediato cuyo valor es 15.
\end{itemize}


\item \textbf{Directo}
Con este modo de direccionamiento se denota un operando alojado en una dirección de memoria o de puertos. La codificación de este modo se indica en la tabla \ref{tablamodos}.

Ejemplos:
\begin{itemize}
\item \textbf{[0x0000]} denota un operando cuyo valor se encuentra en la celda de memoria cuya dirección es \textbf{0x0000}.
\item \textbf{[0x000F]} denota  un operando cuyo valor se encuentra en la celda de memoria cuya dirección es \textbf{0x000F}.
\end{itemize}


\item \textbf{Indirecto}
Con este modo de direccionamiento se denota un operando alojado en una celda de memoria cuya dirección está almacenada en otra celda de memoria. La codificación de este modo se indica en la tabla \ref{tablamodos}.

Ejemplos:
\begin{itemize}
\item \textbf{[[0x0000]]} denota un operando cuyo valor se encuentra en la celda de memoria cuya dirección esta guardada como dato en la celda de memoria cuya dirección es \textbf{0x0000}
\item \textbf{[[0x000F]]} denota un operando cuyo valor se encuentra en la celda de memoria cuya dirección esta guardada como dato en la celda de memoria cuya dirección es \textbf{0x000F}
\end{itemize}

\item \textbf{Registro} Con este modo de direccionamiento se denota un operando alojado en un registro de uso general (R0 a R7). La codificación de este modo se indica en la tabla \ref{tablamodos}.

Ejemplos:
\textbf{R0} denota un operando almacenado en el registro R0. 
\textbf{R7} denota un operando almacenado en el registro R7.

\item \textbf{Registro Indirecto} De manera similar al modo indirecto, con este modo de direccionamiento se denota un operando alojado en una celda de memoria cuya dirección está almacenada en el registro indicado. La codificación de este modo se indica en la tabla \ref{tablamodos}.


Ejemplos:
\textbf{[R0]} denota un operando almacenado en una celda de memoria cuya dirección está en el registro \textbf{R0}.
\textbf{[R7]} denota un operando almacenado en una celda de memoria cuya dirección está en el registro \textbf{R7}.

\end{enumerate}

\tabla{|l|l|l|}{
\textbf{Modo}       &  \textbf{Codificación}\\ \hline\hline
Inmediato   &  000000\\ \hline
Directo     &  001000\\ \hline
Indirecto   &  011000\\ \hline
Registro    &  100rrr\\ \hline
Registro indirecto &  110rrr\\ \hline
}{\label{tablamodos}Tabla de códigos de los modos de direccionamiento (Nota: rrr describe el número de registro)}

%%%%%%%%%%%%%%%%%%%%%%%%%%%%%%%%%%%%%%%%%%%%%%%%%%%%%%%%%%%
\subsection{Repertorio de instrucciones}
En esta sección se detalla cómo se construye el código máquina de las instrucciones de la arquitectura.
%%%%%%%%%%%%%%%%%%%%%%%%%%%%%%%%%%%%%%%%%%%%%%%%%%%%%%%%%%%
\subsubsection{Instrucciones de 2 operandos}

A continuación se muestra la codificación (formato) de las instrucciones de dos operandos:

\formatoinstr{|c|c|c|c|c|c|}{
Codigo de Operación  & \mdest{}& \msrc{}  & \dest{} & \src{}\\
(4b)     &   (6b)  &  (6b)   &  (16b)  &  (16b)} 



Las instrucciones de dos operandos descriptas a continuación son instrucciones aritméticas o lógicas donde se asume que el resultado de la operación se almacena en uno de los dos operandos de entrada, y por lo tanto se lo denomina \textbf{operando destino}.

\begin{enumerate}
\item \textbf{MUL destino, origen}
Código de operación: 0000
Esta instrucción describe la multiplicación entre los datos de los dos operandos. Esta operación es la única que cuyo resultado puede ser 32 bits, que son lo que ocuparía más de una celda de memoria en código binario, por lo que los primeros 16 bits, es decir, la primer mitad, es guardada en el registro \textbf{R7} y la segunda en el operando destino.
 
\item \textbf{ADD destino, origen}
Código de operación: 0010
Esta instrucción describe la suma entre los datos de los dos operandos. El resultado de la ejecución de la suma es guardado en el operando destino.

\item \textbf{SUB destino, origen}
Código de operación: 0011
Esta instrucción describe la resta entre los datos de los dos operandos. El resultado de la ejecución de dicha resta es guardado en el operando destino.

\item \textbf{DIV destino, origen}
Código de operación: 0111
Esta instrucción describe la división entre el dato en el operando destino como dividendo y el dato en el operando origen como divisor. El resultado de la ejecución de la división es guardado en el operando destino.

\item \textbf{MOV destino, origen}
Código de operación: 0001
Esta instrucción describe la copia de datos del dato alojado en el operando origen al operando destino. El resultado de la ejecución del MOV es el dato guardado en el operando origen ahora también guardado en el operando destino.

\item \textbf{AND destino, origen}
Código de operación: 0100
Esta instrucción describe la operación lógica "y" bit a bit entre los datos de los dos operandos. El resultado de la ejecución de esta operación es guardado en el operando destino.

\item \textbf{CMP destino, origen}
Código de operación: 0110
Esta instrucción describe la resta entre dos operandos, sin guardar el resultado. Su único efecto es la actualización de flags en la cpu.

\item \textbf{OR destino, origen}
Código de operación: 0101
Esta instrucción describe la operación lógica "o" bit a bit entre los datos de los dos operandos. El resultado de la ejecución de esta operación es guardado en el operando destino.
\end{enumerate}

%%%%%%%%%%%%%%%%%%%%%%%%%%%%%%%%%%%%%%%%%%%%%%%%%%%%%%%%%%%
\subsubsection{Instrucciones de 1 operando origen  (falta revisar Mara)}

El formato de instrucción de un operando origen es el siguiente:

  CodOp   +   relleno   +  Modo origen +   Origen
(4 bits)      (000000)        (6 bits)    (16 bits)

\begin{enumerate}
\item \textbf{CALL origen}
Código de operación: 1011
El efecto del CALL es guardar la dirección de memoria en la celda de la dirección que se encuentra guardada en el SP (Stack pointer) aumentar el SP y guardar en el PC (Program Counter) el dato que se encuentra guardado en el operando origen ya que describe el llamado a una subrutina que comienza en la celda de memoria cuya dirección esta guardada en el operando origen.

\item \textbf{JMP origen}
Código de operación: 0110
El efecto del JMP es cambiar el PC (Program Counter) por el dato que esta guardado en el operando origen ya que esta operación describe el salto a otra parte de la memoria para continuar con la ejecución del programa.
\end{enumerate}

%%%%%%%%%%%%%%%%%%%%%%%%%%%%%%%%%%%%%%%%%%%%%%%%%%%%%%%%%%%
\subsubsection{Instrucciones de 1 operando destino  (falta revisar Mara)}

El formato de instrucción de un operando destino es el siguiente:

  CodOp   +  Modo origen  +  relleno  +  Origen
(4 bits)      (6 bits)      (000000)    (16 bits)

\begin{enumerate}
\item \textbf{NOT destino}
Código de operación: 1001
Esta instrucción describe la operación lógica "negación" bit a bit en el datos del operando destino. El resultado de la ejecución de esta operación es guardado en la misma celda o registro de donde es leído el dato inicialmente.
\end{enumerate}

%%%%%%%%%%%%%%%%%%%%%%%%%%%%%%%%%%%%%%%%%%%%%%%%%%%%%%%%%%%
\subsubsection{Instrucciones sin operandos  (falta revisar Mara)}

El formato de instrucción sin operandos es el siguiente:

 CodOp     +    relleno 
(4 bits)     (000000000000)


\begin{enumerate}
\item \textbf{RET}
Código de operación: 0110
El efecto del ret es cambiar el pc por el dato que esta guardado en la celda de memoria que se encuentra en el SP (Stack pointer) y decrementar el SP ya que describe la finalización de la ejecución de una subrutina y la ejecución del resto del programa.
\end{enumerate}


%%%%%%%%%%%%%%%%%%%%%%%%%%%%%%%%%%%%%%%%%%%%%%%%%%%%%%%%%%%
\subsubsection{Instrucciones de salto condicional  (falta revisar Mara)}

El formato de instrucción de salto condicional es el siguiente es el siguiente:

 prefijo +   CodOp   +  desplazamiento 
 (1111)     (4 bits)     (8 bits)

El efecto de cualquier salto condicional es aumentar el PC (Program Counter) en la cantidad de celdas que indique el desplazamiento si sólo la condición que cada salto condicional tiene da como resultado 1, lo cual es interpretado como verdadero.

\begin{enumerate}
\item \textbf{JE desplazamiento}
Código de operación: 0001
La condición del salto es que el flag \textbf{Z} (Cero) sea 1, es decir la ultima operación matemática dió como resultado el número cero.

\item \textbf{JNE desplazamiento}
Código de operación: 1001
La condición del salto es que el flag \textbf{Z} (Cero) sea 0, es decir la ultima operación matemática no dió como resultado el número cero.

\item \textbf{JLE desplazamiento}
Código de operación: 0010
La condición del salto es el resultado de la siguiente operación lógica \textbf{Z OR ( N XOR V )}, es decir la ultima operación matemática es menor o igual con signo.

\item \textbf{JG desplazamiento}
Código de operación: 1010
La condición del salto es el resultado de la siguiente operación lógica \textbf{NOT (Z OR ( N XOR V ))}, es decir la ultima operación matemática es mayor con signo.

\item \textbf{JL desplazamiento}
Código de operación: 0011
La condición del salto es el resultado de la siguiente operación lógica \textbf{N XOR V}, es decir la ultima operación matemática es menor con signo.

\item \textbf{JGE desplazamiento}
Código de operación: 1011
La condición del salto es el resultado de la siguiente operación lógica \textbf{NOT (N XOR V)}, es decir la ultima operación matemática es mayor o igual con signo.

\item \textbf{JLEU desplazamiento}
Código de operación: 0100
La condición del salto es el resultado de la siguiente operación lógica \textbf{C OR Z}, es decir la ultima operación matemática es menor o igual sin signo.

\item \textbf{JGU desplazamiento}
Código de operación: 1100
La condición del salto es el resultado de la siguiente operación lógica \textbf{NOT (C OR Z)}, es decir la ultima operación matemática es mayor sin signo.

\item \textbf{JCS desplazamiento}
Código de operación: 0101
La condición del salto es que el flag \textbf{C} sea 1, es decir la ultima operación matemática es menor sin signo.

\item \textbf{JNEG desplazamiento}
Código de operación: 0101
La condición del salto es que el flag \textbf{N} sea 1, es decir si el último resultado de una operación dio negativo.

\item \textbf{JVS desplazamiento}
Código de operación: 0111
La condición del salto es que el flag \textbf{V} sea 1, es decir si el último resultado de una operación dio overflow.

\end{enumerate}
\section{Como utilizar el simulador}

\verde{En esta sección debemos mostrar como se arranca la aplicación y como se carga un programa .qsim con algunos pantallazos}

https://github.com/molinarirosito/QSim

https://github.com/molinarirosito/QSim-UI



%%%%%%%%%%%%%%%%%%%%%%%%%%%%%%%%%%%%%%%%%%%%%%%%%%%%
\begin{thebibliography}{9}

\bibitem{Stallings} Williams Stallings, \emph{Computer Organization and Architecture}, octava edición, Editorial Prentice Hall, 2010.

\bibitem{Tanenbaum} A. Tanenbaum, \emph{Organización de Computadoras}, cuarta edición, Editorial Pearson.

\bibitem{Patterson} Hennessy, Patterson. \emph{Arquitectura de Computadores - Un enfoque cuantitativo}, primera edición,  Editorial Mc Graw Hill.

\bibitem{blog} Sitio oficial de la materia \orga: \url{http:\\orga.blog.unq.edu.ar} (2013)

\end{thebibliography}

\end{document}